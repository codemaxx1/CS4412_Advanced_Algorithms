\documentclass[12pt, a4paper]{article}
\addtolength{\oddsidemargin}{-.875in}
\addtolength{\evensidemargin}{-.875in}
\addtolength{\textwidth}{1.75in}
\addtolength{\topmargin}{-.875in}
\addtolength{\textheight}{1.75in}

\usepackage{indentfirst}
\usepackage{graphicx}
\usepackage{amsmath}


\begin{document}
\noindent
Nicholas Garrett\\ \\
Professor Beard\\ \\
CS 4412\\ \\
11/17/2021\\ \\


\begin{center}
	\centering{	B+ Tree\\ }
\end{center}


The interface will be simple, using a similar output system as that of a binary tree.  The nodes will be printed as sections.  

\begin{verbatim}
For example: " 
									| <min key> <key> <key> <key> <maxkey>|		(root)
			| <min key> <key> <key> <key> <max key>| 				| <min key> <key> <key> <key> <max key>|
"
\end{verbatim}

\end{document}  